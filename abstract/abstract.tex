%%%%%%%%%%%%%%%%%%%%%%%%%%%%%%%%%%%%%%%%%
% Journal Article
% LaTeX Template
% Version 2.1 (Jan 18, 2024)
%
% This template originates from:
% https://www.LaTeXTemplates.com
%
% Author:
% Vel (vel@latextemplates.com)
%
% License:
% CC BY-NC-SA 4.0 (https://creativecommons.org/licenses/by-nc-sa/4.0/)
%
% NOTE: The bibliography needs to be compiled using the biber engine.
%
%%%%%%%%%%%%%%%%%%%%%%%%%%%%%%%%%%%%%%%%%

%----------------------------------------------------------------------------------------
%	PACKAGES AND OTHER DOCUMENT CONFIGURATIONS
%----------------------------------------------------------------------------------------

\documentclass[
	a4paper, % Paper size, use either a4paper or letterpaper
	10pt, % Default font size, can also use 11pt or 12pt, although this is not recommended
	unnumberedsections, % Comment to enable section numbering
	twoside, % Two side traditional mode where headers and footers change between odd and even pages, comment this option to make them fixed
]{LTJournalArticle}

\addbibresource{bibliography.bib} % BibLaTeX bibliography file
\renewcommand*{\bibfont}{\footnotesize}

\runninghead{} % A shortened article title to appear in the running head, leave this command empty for no running head

\footertext{\textit{RISC-V Summit Europe, Paris, 12-15th May 2025}} % Text to appear in the footer, leave this command empty for no footer text

\setcounter{page}{1} % The page number of the first page, set this to a higher number if the article is to be part of an issue or larger work

%----------------------------------------------------------------------------------------
%	TITLE SECTION
%----------------------------------------------------------------------------------------

\title{Enabling RISC-V CI in Open-Source Projects: Challenges and Solutions}
% Article title, use manual lines breaks (\\) to beautify the layout

% Authors are listed in a comma-separated list with superscript numbers indicating affiliations
% \thanks{} is used for any text that should be placed in a footnote on the first page, such as the corresponding author's email, journal acceptance dates, a copyright/license notice, keywords, etc
\author{%
	% (redacted)\textsuperscript{1}
	Marek Pikuła\textsuperscript{1}
}

% Affiliations are output in the \date{} command
% \date{\footnotesize\textsuperscript{\textbf{1}}(redacted)}
\date{\footnotesize\textsuperscript{\textbf{1}}(Samsung R\&D Institute Poland)}

% Full-width abstract
\renewcommand{\maketitlehookd}{%
	\begin{abstract}
		\noindent The adoption of RISC-V as a viable architecture for open-source software development is gaining traction. However, a major challenge remains: ensuring continuous integration (CI) support for RISC-V in upstream projects. At Samsung R\&D Institute Poland, we addressed this issue by enabling RISC-V CI for several Freedesktop.org (FDO) projects, including Pixman and GStreamer Orc, and we are currently extending our work to the Opus codec. This work presents our approach to enabling RISC-V CI in FDO projects, addressing the challenges of testing architecture-specific optimizations without native hardware support. We detail our implementation of Docker-based GitLab runners with QEMU emulation, enabling automated multi-architecture testing while minimizing infrastructure overhead. Our work not only enhances software quality by enabling automated testing for RISC-V but also provides a framework for future contributions to seamlessly integrate RISC-V into open-source CI ecosystems.
	\end{abstract}
}

%----------------------------------------------------------------------------------------

\begin{document}

\maketitle % Output the title section

%----------------------------------------------------------------------------------------
%	ARTICLE CONTENTS
%----------------------------------------------------------------------------------------

\section{Introduction}

The RISC-V ecosystem has seen rapid expansion, yet upstream open-source projects often lack Continuous Integration (CI) support for this architecture. Many projects primarily test on x86 and ARM due to limited infrastructure availability. This lack of CI resources prevents maintainers from validating RISC-V-specific optimizations, creating a barrier to broader adoption.

This limitation became evident when integrating RISC-V Vector extension (RVV) support into the Pixman library. Without access to RISC-V hardware, maintainers faced difficulties in validating the implementation. To address this, we collaborated with Freedesktop.org (FDO) CI maintainers to develop an efficient testing workflow that verifies architecture-specific backends without overwhelming limited CI resources. We are also in the process of extending our approach to additional projects, including GStreamer Orc and Opus.

\section{Overview of Targeted Libraries}

\subsection{Pixman}

Pixman\autocite{pixman} is a low-level pixel manipulation library used by various graphics stacks, including the X server and Cairo. It features architecture-specific SIMD backends for optimized performance on different platforms such as x86 (MMX/SSE2, SSSE3), ARM (NEON), MIPS (DSPr2), PowerPC (AltiVec/VMX), and LoongArch (MMI). SIMD (Single Instruction, Multiple Data) is a parallel computing technique that accelerates processing by performing the same operation on multiple data points simultaneously.

\subsection{GStreamer ORC}

GStreamer ORC (The OIL Runtime Compiler)\autocite{orc} is a library that dynamically generates vectorized code from ORC bytecode, enhancing media processing performance. Similar to Pixman, it contains multiple architecture-specific backends.

\subsection{Opus}

Opus\autocite{rfc6716} is a widely adopted audio codec designed for high-quality, low-latency voice and music streaming. Its architecture-specific optimizations, providing benefits in the processing speed and energy efficiency, make it a prime candidate for RISC-V enablement, particularly with RVV support.

\section{Creating the CI Workflow}

FDO's GitLab CI infrastructure is limited to x86 and ARM, as it relies on donated computing resources. To enable RISC-V testing without adding significant maintenance overhead, we implemented a solution using Docker-based GitLab runners with QEMU user-mode emulation\autocite{dockerqemu}.

Our primary objectives were to:
\begin{itemize}
      \item Provide CI coverage for all supported architectures (including basic support for Windows using Wine).
      \item Use native ARM runners where applicable.
      \item Build the libraries with both GNU and LLVM toolchains.
      \item Execute tests for all SIMD backends.
      \item Generate a consolidated coverage report.
\end{itemize}

\subsection{Docker Image Preparation}

We developed reusable GitLab CI templates to build multi-architecture Docker images, leveraging extensive Debian's cross-architecture support and pre-built base images.

To enable comprehensive testing, our images included necessary toolchains (GNU, LLVM), required library dependencies, and tooling for analyzing test results.

Using a multi-stage, composable Docker image approach, we ensured flexibility—allowing projects to either use pre-built CI images (GNU, LLVM, Meson stack) or extend them for project-specific needs (e.g., by installing required system dependencies).

Whenever possible, we prioritized native execution to simplify debugging and enable coverage analysis. However, for certain architectures (e.g., big-endian PowerPC 32-bit), cross-compilation was required, in which case we focused solely on correctness verification.

\subsection{Build Stage}

Initially, Pixman's CI workflow only ran tests with the GNU toolchain, leading to overlooked issues in LLVM builds. Architecture-specific code (e.g., with compiler intrinsics) can behave differently between compilers, necessitating dual-toolchain testing. Including both GNU and LLVM in our workflow uncovered several previously unreported LLVM-related bugs, improving the library's overall reliability.

\subsection{Test Stage}

The CI workflow executes tests across all supported architecture-specific backends. For instance, x86 builds reuse binaries across MMX, SSE2, and SSSE3 SIMD implementations allowing for running the tests in parallel.

Similarly, for RVV, we tested multiple vector register lengths (VLENs) to identify configuration-specific issues. Since development hardware we used supported a VLEN of 256, automated testing of various configurations ensured early discovery of potential issues for future RISC-V targets.

\subsection{Summary Stage}

Native execution simplified coverage analysis, but each test run generated separate reports. Manually reviewing over 40 reports was impractical, so we used \emph{gcovr}\autocite{gcovr} to merge them into a unified summary, producing both human-readable and machine-consumable reports for GitLab’s CI system\autocite{gitlab-cov}.

\section{Impact and Future Work}

Our methodology has already uncovered early-stage bugs in Pixman's RVV implementation by testing across various VLEN configurations. The same approach will be extended to GStreamer Orc to enhance RISC-V support in other media processing workloads.

For Opus, automated RISC-V testing significantly accelerates development, reducing reliance on manual hardware validation. Our work establishes a scalable model for integrating RISC-V into CI workflows.

Looking ahead, we plan to expand our methodology to additional projects and collaborate with open-source communities to further enhance RISC-V CI capabilities.

It's worth to mention the Cloud-V CI/CD system\autocite{cloudv}, which provides real, hardware RISC-V targets for use in CI/CD workflows. This looks like an interesting option for the future, in case a project requires, e.g., performance measurement in CI, which cannot be provided in QEMU environment.

\section{Conclusion}

By enabling RISC-V CI in upstream open-source projects, we have addressed one of the key barriers to RISC-V adoption. Our approach is generic enough that it can be reused in other Freedesktop.org projects and in external GitLab instances used by other open source projects. This work ensures that maintainers can validate RISC-V contributions without requiring dedicated hardware, making it easier for developers to contribute RISC-V optimizations. This success story demonstrates how strategic CI integration can accelerate the growth of RISC-V within the open-source ecosystem.

%----------------------------------------------------------------------------------------
%	 REFERENCES
%----------------------------------------------------------------------------------------

\printbibliography % Output the bibliography

% \subsection{Acknowledgments}

% We extend our gratitude to the Freedesktop.org maintainers for their support in integrating RISC-V CI workflows.

%----------------------------------------------------------------------------------------

\end{document}
